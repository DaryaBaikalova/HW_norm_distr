
%%%%%%%%%%%%%%%%%%%%%%%%%%%%%%%%%%%%%%%%%
% Dreuw & Deselaer's Poster
% LaTeX Template
% Version 1.0 (11/04/13)
%
% Created by:
% Philippe Dreuw and Thomas Deselaers
% http://www-i6.informatik.rwth-aachen.de/~dreuw/latexbeamerposter.php
%
% This template has been downloaded from:
% http://www.LaTeXTemplates.com
%
% License:
% CC BY-NC-SA 3.0 (http://creativecommons.org/licenses/by-nc-sa/3.0/)
%
%%%%%%%%%%%%%%%%%%%%%%%%%%%%%%%%%%%%%%%%%

%----------------------------------------------------------------------------------------
%	PACKAGES AND OTHER DOCUMENT CONFIGURATIONS
%----------------------------------------------------------------------------------------

\documentclass[final,hyperref={pdfpagelabels=false}]{beamer}

\usepackage[orientation=portrait,size=a1,scale=1]{beamerposter} % ЗДЕСЬ МЕНЯТЬ РАЗМЕР ПЛаката

\usetheme{GeoURV} % Use the GeoURV theme supplied with this template

\usepackage[english, catalan]{babel} % English language/hyphenation

\usepackage{apacite}
\usepackage{natbib}
\usepackage{amsmath,amsthm,amssymb,latexsym} % For including math equations, theorems, symbols, etc
\usepackage{lipsum}

%\usepackage{times}\usefonttheme{professionalfonts}  % Uncomment to use Times as the main font
%\usefonttheme[onlymath]{serif} % Uncomment to use a Serif font within math environments

%\boldmath 
% Use bold for everything within the math environment

\usepackage{booktabs} % Top and bottom rules for tables

\graphicspath{{figures/}} % Location of the graphics files



\title{\huge Вывод Нормального Распределения из Предпосылок}
\author{Дарья Байкалова, Евгения Никонова, Арина Петросян}
\institute{НИУ ВШЭ}

\newcommand{\leftfoot}{}
\newcommand{\rightfoot}{}

\usepackage[T2A]{fontenc}
\usepackage[utf8]{inputenc}
\usepackage[russian]{babel}
\usepackage{tikz}
\usetikzlibrary{trees}
\usepackage[left=3cm,right=3cm,top=1cm,bottom=3cm,bindingoffset=0cm]{geometry}
\usepackage{listings}
\usepackage{hyperref}
\usepackage{mathrsfs}
\usepackage{tcolorbox}
\usepackage{icomma} 
\usepackage{pgfplots}
\usepackage{multicol}
\usetikzlibrary{patterns}
\usepackage{relsize} % чтобы увеличить размеры всего в несколько раз
\usetikzlibrary{backgrounds}

\DeclareMathOperator{\Cov}{Cov}
\DeclareMathOperator{\Corr}{Corr}
\DeclareMathOperator{\Var}{Var}
\DeclareMathOperator{\E}{E}


\begin{document}
\relscale{1.3} % увеличиваем текст в 1.3 раза


\begin{frame}[t] % The whole poster is enclosed in one beamer frame

\begin{columns}[t] % The whole poster consists of two major columns, each of which can be subdivided further with another \begin{columns} block - the [t] argument aligns each column's content to the top

\begin{column}{.02\textwidth}\end{column} % Empty spacer column

\begin{column}{.465\textwidth} % The first column

%------------
% Предпосылки
%------------
\begin{block}{\large{Предпосылки}}
\begin{enumerate}
\begin{Large}
		\item Закон распределения вектора инвариантен к повороту
		\item Проекции вектора на ортогональные подпространства независимы
		\item Функция совместной плотности непрерывна
\end{Large}
	\end{enumerate}
\end{block}




%--------------
% INTRODUCCIÓ
%--------------
\begin{block}{\large{Вывод нормального распределения}}

\begin{minipage}{.4\textwidth}
		Пусть $ X, Y $ - случайные величины\\
		\begin{itemize}
		\item $ \left(X; Y \right)_e $ - вектор в базисе $ \left\lbrace e_1, e_2 \right\rbrace  $
		\item $ \left(Y; -X \right)_f $ - вектор в базисе $ \left\lbrace f_1, f_2 \right\rbrace  $ 
		\end{itemize}
\end{minipage}
    \begin{minipage}{.5\textwidth}
	\begin{tikzpicture}[scale=0.3]
	\draw[<->] (0,15)node[above] {$e_2$} -- (0,0) -- (15,0)node[below] {$e_1$};
	\draw[<->] (40,15)node[above] {$f_1=e_2$} -- (40,0) -- (25,0)node[below] {$f_2=-e_1$};
	\end{tikzpicture} 
\end{minipage}
\vspace{10mm}

Будем считать, что существуют $\E(X), \E(Y), \Var(X), \Var(Y)$, тогда:\\
\begin{minipage}{.3\textwidth}
	
		\[
		\begin{cases}
		\E(X)=\E(Y)\\
		\E(X)=-\E(Y)
		\end{cases}
		\]
		
\end{minipage}
\begin{minipage}{.7\textwidth}
	\[
	\Rightarrow \quad \E(X)=-\E(X) \quad \Rightarrow \quad \E(X)=\E(Y)=0; \quad \Var(X)=\Var(Y)
	\]
\end{minipage}

\vspace{0.1 in} 
\vspace{15mm}
Из 2-ой предпосылки следует, что совместная функция плотности случайных величин $X, Y$ представима в виде:
\[f_{X,Y}=f_X(x)f_Y(y)\]
\vspace{5mm}

В силу инвариантности к повороту (1-ая предпосылка) функция плотности зависит только от длины вектора $(X, Y)^T$:
\[f_X(x)=f_Y(x) \quad \Rightarrow \quad X \sim Y \sim (-X)  \]
\[\Rightarrow \quad f_X(x) \quad \text{симметрична относительно нуля} \Rightarrow \quad f_X(x)=g(x^2) \]
\vspace{0.05 in} 
\\
Пусть $f_{XY}(x,y)=h(x^2+y^2)=g(x^2)g(y^2)$, рассмотрим $h(x^2+y^2)$:
\[ \ln h(x^2+y^2)= \ln g(x^2)+\ln g(y^2)\]
\[\ln h(x^2) = \ln g(x^2) + C, \quad \text{где} \quad C=\ln g(0)-\text{const} \]
\[ \ln g(x^2) = \ln h(x^2) - C\]
\vspace{0.05 in} 
\\
Заметим, что:
\[ \ln g(nx^2) = \ln h(nx^2) - C= \ln h ((n-1)x^2+x^2) - C = \ln g ((n-1)x^2)+  \ln g(x^2) - C =\]
\[= \ln h((n-1)x^2) - C + \ln g(x^2) - C = \ln g((n-2)x^2) + 2\ln g(x^2) - 2C = ...=\]
\[=\ln g(0) + n\ln g(x^2)-nC=n\ln g(x^2)-(n-1)C\]
\vspace{0.05 in} 
\\
Пусть $x^2=a$, тогда функция $\tilde{g}(a)=\ln g(a)$ обладает свойством:
\[\tilde{g}(na)=n\tilde{g}(a)-(n-1)C\]
Тогда:
\[\tilde{g}(na)-C=n\tilde{g}(a)-nC=n(\tilde{g}(a)-C)\]
\vspace{0.05 in} 
\\
Предположим, что:
\[\tilde{g}(a)=\ln g(a) =ka + C\]
\vspace{0.05 in} 
\\
Покажем, что такая функция удовлетворяет свойствам $\tilde{g}(a)$:
\[\tilde{g}(na)=nka+C=nka+nC-(n-1)C=n(ka-C)-(n-1)C=n\tilde{g}(a)-(n-1)C\]
\vspace{0.05 in} 
\\

Тогда:
\[ \ln g(x^2) = kx^2+C\]
\[g(x^2)=e^{kx^2+c}=e^{kx^2}e^C\]
Значит,
\[f_X(x)=e^{kx^2}e^C\]
\vspace{0.05 in} 
\\
Из свойств функции плотности найдём $C$ и $k$:
\[ \int_{-\infty}^{\infty} e^{kx^2}e^C dx= 1\]
\[e^C   \int_{-\infty}^{\infty} e^{kx^2} dx= e^C \frac{\sqrt{\pi}}{\sqrt{-k}}=1 \]
\[e^C = \frac{\sqrt{-k}}{\sqrt{\pi}}\]
\[f_X(x)=\frac{\sqrt{-k}}{\sqrt{\pi}} e^{kx^2}, k<0\]
Сделаем замену $\lambda = \frac{\sqrt{-k}}{\sqrt{\pi}}$, тогда $f_X(x)=\lambda e^{-\pi{\lambda}^2x^2}$
\vspace{0.4 in} 
\\






\end{block}

\end{column} % End of the first column

\begin{column}{.03\textwidth}\end{column} % Empty column

\begin{column}{.465\textwidth} % The second column
%---------
% RESULTS
%---------


%\vspace{1cm}
%\begin{center}
%    \textbf{График зависимости $f_X(x) = \lambda e^{-\pi \lambda^2 x^2}$ от %$\lambda$ }
%\end{center}
%\vspace{3cm}

\definecolor{beaublue}{rgb}{1.0, 1.0, 1.0}
\begin{tcolorbox}[colback = beaublue]
\begin{center}
  

\textbf{График зависимости $f_X(x) = \lambda e^{-\pi \lambda^2 x^2}$ от $\lambda$ }
\end{center}
\end{tcolorbox}

\vspace{2.5cm}
\begin{tikzpicture}
	% This file was created by tikzplotlib v0.9.2.
\begin{tikzpicture}

\definecolor{color0}{rgb}{0.886274509803922,0.290196078431373,0.2}
\definecolor{color1}{rgb}{0.203921568627451,0.541176470588235,0.741176470588235}
\definecolor{color2}{rgb}{0.596078431372549,0.556862745098039,0.835294117647059}

\begin{axis}[
axis background/.style={fill=white!89.8039215686275!black},
axis line style={white},
legend cell align={left},
legend style={fill opacity=0.8, draw opacity=1, text opacity=1, draw=white!80!black, fill=white!89.8039215686275!black},
tick align=outside,
tick pos=left,
title={\(\displaystyle f_X(x) = \lambda e^{-\pi \lambda^2 x^2}\) },
x grid style={white},
xlabel={x},
xmajorgrids,
xmin=-1.1, xmax=1.1,
xtick style={color=white!33.3333333333333!black},
y grid style={white},
ylabel={\(\displaystyle f_X(x)\)},
ymajorgrids,
ymin=-0.2, ymax=4.2,
ytick style={color=white!33.3333333333333!black}
]
\addplot [semithick, color0]
table {%
-1 0.0432139182637723
-0.983870967741935 0.0477838182636073
-0.967741935483871 0.052750694352476
-0.951612903225806 0.0581387430216265
-0.935483870967742 0.0639724851387078
-0.919354838709677 0.0702766308951519
-0.903225806451613 0.0770759294685356
-0.887096774193548 0.0843950034119078
-0.870967741935484 0.0922581679661766
-0.854838709677419 0.100689235691206
-0.838709677419355 0.109711307024694
-0.82258064516129 0.119346547603274
-0.806451612903226 0.129615953415197
-0.790322580645161 0.140539105095798
-0.774193548387097 0.152133912922508
-0.758064516129032 0.16441635431215
-0.741935483870968 0.177400205865804
-0.725806451612903 0.191096772241748
-0.709677419354839 0.205514614360584
-0.693548387096774 0.220659279654366
-0.67741935483871 0.236533037258823
-0.661290322580645 0.253134621210309
-0.645161290322581 0.270458984842387
-0.629032258064516 0.288497069676893
-0.612903225806452 0.307235592166929
-0.596774193548387 0.326656851670887
-0.580645161290323 0.346738563014187
-0.564516129032258 0.367453716926238
-0.548387096774194 0.388770471522247
-0.532258064516129 0.410652077831473
-0.516129032258065 0.433056842154873
-0.5 0.455938127765996
-0.483870967741935 0.479244398150545
-0.467741935483871 0.502919303614262
-0.451612903225806 0.526901812678595
-0.435483870967742 0.551126389232711
-0.419354838709677 0.57552321592359
-0.403225806451613 0.60001846374863
-0.387096774193548 0.624534607273712
-0.370967741935484 0.648990784341037
-0.354838709677419 0.673303198562848
-0.338709677419355 0.697385562327494
-0.32258064516129 0.72114957748163
-0.306451612903226 0.744505450305449
-0.290322580645161 0.767362436875349
-0.274193548387097 0.789629414419236
-0.258064516129032 0.811215473822028
-0.241935483870968 0.832030528041063
-0.225806451612903 0.851985930850402
-0.209677419354839 0.870995100056373
-0.193548387096774 0.888974139119839
-0.17741935483871 0.905842450988788
-0.161290322580645 0.921523337891439
-0.145161290322581 0.93594458086806
-0.129032258064516 0.949038992930155
-0.112903225806452 0.960744939928786
-0.0967741935483871 0.971006823488082
-0.0806451612903226 0.979775520712821
-0.0645161290322581 0.987008775806136
-0.0483870967741936 0.992671539229824
-0.032258064516129 0.99673625059859
-0.0161290322580645 0.999183062113485
0 1
0.0161290322580645 0.999183062113485
0.032258064516129 0.99673625059859
0.0483870967741935 0.992671539229824
0.064516129032258 0.987008775806136
0.0806451612903225 0.979775520712821
0.096774193548387 0.971006823488082
0.112903225806452 0.960744939928786
0.129032258064516 0.949038992930156
0.145161290322581 0.93594458086806
0.161290322580645 0.921523337891439
0.17741935483871 0.905842450988788
0.193548387096774 0.888974139119839
0.209677419354839 0.870995100056373
0.225806451612903 0.851985930850402
0.241935483870968 0.832030528041063
0.258064516129032 0.811215473822028
0.274193548387097 0.789629414419236
0.290322580645161 0.767362436875349
0.306451612903226 0.744505450305449
0.32258064516129 0.721149577481631
0.338709677419355 0.697385562327494
0.354838709677419 0.673303198562849
0.370967741935484 0.648990784341037
0.387096774193548 0.624534607273712
0.403225806451613 0.60001846374863
0.419354838709677 0.57552321592359
0.435483870967742 0.551126389232711
0.451612903225806 0.526901812678595
0.467741935483871 0.502919303614262
0.483870967741935 0.479244398150545
0.5 0.455938127765996
0.516129032258065 0.433056842154873
0.532258064516129 0.410652077831473
0.548387096774194 0.388770471522247
0.564516129032258 0.367453716926238
0.580645161290323 0.346738563014187
0.596774193548387 0.326656851670888
0.612903225806452 0.307235592166929
0.629032258064516 0.288497069676893
0.645161290322581 0.270458984842388
0.661290322580645 0.253134621210309
0.67741935483871 0.236533037258823
0.693548387096774 0.220659279654366
0.709677419354839 0.205514614360584
0.725806451612903 0.191096772241748
0.741935483870968 0.177400205865804
0.758064516129032 0.16441635431215
0.774193548387097 0.152133912922508
0.790322580645161 0.140539105095798
0.806451612903226 0.129615953415197
0.82258064516129 0.119346547603274
0.838709677419355 0.109711307024694
0.854838709677419 0.100689235691206
0.870967741935484 0.0922581679661766
0.887096774193548 0.0843950034119079
0.903225806451613 0.0770759294685357
0.919354838709677 0.0702766308951519
0.935483870967742 0.0639724851387078
0.951612903225806 0.0581387430216265
0.967741935483871 0.052750694352476
0.983870967741935 0.0477838182636073
1 0.0432139182637723
};
\addlegendentry{$\lambda$ = 1}
\addplot [semithick, color1]
table {%
-1 6.97468471241799e-06
-0.983870967741935 1.04268564766767e-05
-0.967741935483871 1.54861234863173e-05
-0.951612903225806 2.28503337363588e-05
-0.935483870967742 3.34967663543335e-05
-0.919354838709677 4.87835866922419e-05
-0.903225806451613 7.0583806922635e-05
-0.887096774193548 0.000101460485460076
-0.870967741935484 0.000144893632629433
-0.854838709677419 0.000205571153416382
-0.838709677419355 0.000289758077535054
-0.82258064516129 0.000405760192175101
-0.806451612903226 0.000564499849692395
-0.790322580645161 0.000780222967904597
-0.774193548387097 0.00107135681849843
-0.758064516129032 0.0014615378005486
-0.741935483870968 0.00198082666461791
-0.725806451612903 0.00266712519503643
-0.709677419354839 0.00356780276272249
-0.693548387096774 0.00474153302731743
-0.67741935483871 0.00626033004159915
-0.661290322580645 0.00821175883261886
-0.645161290322581 0.0107012780897338
-0.629032258064516 0.0138546519735358
-0.612903225806452 0.0178203446331072
-0.596774193548387 0.0227717854678976
-0.580645161290323 0.028909366550767
-0.564516129032258 0.0364620073934791
-0.548387096774194 0.0456880982355203
-0.532258064516129 0.0568756134841122
-0.516129032258065 0.0703411743257049
-0.5 0.0864278365275445
-0.483870967741935 0.105501388704952
-0.467741935483871 0.127944970277416
-0.451612903225806 0.154151858937071
-0.435483870967742 0.184516335932353
-0.419354838709677 0.219422614049388
-0.403225806451613 0.259231906830395
-0.387096774193548 0.304267825845017
-0.370967741935484 0.354800411731608
-0.354838709677419 0.411029228721169
-0.338709677419355 0.473066074517646
-0.32258064516129 0.540917969684775
-0.306451612903226 0.614471184333858
-0.290322580645161 0.693477126028374
-0.274193548387097 0.777540943044494
-0.258064516129032 0.866113684309745
-0.241935483870968 0.958488796301089
-0.225806451612903 1.05380362535719
-0.209677419354839 1.15104643184718
-0.193548387096774 1.24906921454742
-0.17741935483871 1.3466063971257
-0.161290322580645 1.44229915496326
-0.145161290322581 1.5347248737507
-0.129032258064516 1.62243094764406
-0.112903225806452 1.7039718617008
-0.0967741935483871 1.77794827823968
-0.0806451612903226 1.84304667578288
-0.0645161290322581 1.89807798586031
-0.0483870967741936 1.94201364697616
-0.032258064516129 1.97401755161227
-0.0161290322580645 1.99347250119718
0 2
0.0161290322580645 1.99347250119718
0.032258064516129 1.97401755161227
0.0483870967741935 1.94201364697616
0.064516129032258 1.89807798586031
0.0806451612903225 1.84304667578288
0.096774193548387 1.77794827823968
0.112903225806452 1.70397186170081
0.129032258064516 1.62243094764406
0.145161290322581 1.5347248737507
0.161290322580645 1.44229915496326
0.17741935483871 1.3466063971257
0.193548387096774 1.24906921454742
0.209677419354839 1.15104643184718
0.225806451612903 1.05380362535719
0.241935483870968 0.958488796301089
0.258064516129032 0.866113684309745
0.274193548387097 0.777540943044494
0.290322580645161 0.693477126028374
0.306451612903226 0.614471184333859
0.32258064516129 0.540917969684775
0.338709677419355 0.473066074517646
0.354838709677419 0.411029228721169
0.370967741935484 0.354800411731609
0.387096774193548 0.304267825845017
0.403225806451613 0.259231906830395
0.419354838709677 0.219422614049388
0.435483870967742 0.184516335932353
0.451612903225806 0.154151858937071
0.467741935483871 0.127944970277416
0.483870967741935 0.105501388704952
0.5 0.0864278365275445
0.516129032258065 0.0703411743257049
0.532258064516129 0.0568756134841122
0.548387096774194 0.0456880982355203
0.564516129032258 0.0364620073934791
0.580645161290323 0.0289093665507671
0.596774193548387 0.0227717854678977
0.612903225806452 0.0178203446331072
0.629032258064516 0.0138546519735359
0.645161290322581 0.0107012780897338
0.661290322580645 0.00821175883261889
0.67741935483871 0.00626033004159917
0.693548387096774 0.00474153302731743
0.709677419354839 0.00356780276272249
0.725806451612903 0.00266712519503643
0.741935483870968 0.00198082666461791
0.758064516129032 0.0014615378005486
0.774193548387097 0.00107135681849843
0.790322580645161 0.000780222967904597
0.806451612903226 0.000564499849692395
0.82258064516129 0.000405760192175101
0.838709677419355 0.000289758077535055
0.854838709677419 0.000205571153416382
0.870967741935484 0.000144893632629433
0.887096774193548 0.000101460485460077
0.903225806451613 7.05838069226352e-05
0.919354838709677 4.8783586692242e-05
0.935483870967742 3.34967663543335e-05
0.951612903225806 2.28503337363588e-05
0.967741935483871 1.54861234863173e-05
0.983870967741935 1.04268564766767e-05
1 6.97468471241799e-06
};
\addlegendentry{$\lambda$ = 2}
\addplot [semithick, color2]
table {%
-1 1.57664555280194e-12
-0.983870967741935 3.89626874434316e-12
-0.967741935483871 9.48800445603038e-12
-0.951612903225806 2.27673244330885e-11
-0.935483870967742 5.38344466533261e-11
-0.919354838709677 1.25435264960147e-10
-0.903225806451613 2.87998476594065e-10
-0.887096774193548 6.51586188929296e-10
-0.870967741935484 1.45266238519862e-09
-0.854838709677419 3.19130750008354e-09
-0.838709677419355 6.90849989736984e-09
-0.82258064516129 1.47370310643494e-08
-0.806451612903226 3.09775725928519e-08
-0.790322580645161 6.41646616757949e-08
-0.774193548387097 1.30965098794868e-07
-0.758064516129032 2.63406435887539e-07
-0.741935483870968 5.22045468136291e-07
-0.725806451612903 1.01953330858841e-06
-0.709677419354839 1.96203002335803e-06
-0.693548387096774 3.72066885997466e-06
-0.67741935483871 6.95260459480276e-06
-0.661290322580645 1.28022169803388e-05
-0.645161290322581 2.32291852294759e-05
-0.629032258064516 4.15330582707992e-05
-0.612903225806452 7.31753800383626e-05
-0.596774193548387 0.000127041965076447
-0.580645161290323 0.000217340448944149
-0.564516129032258 0.000366391219863391
-0.548387096774194 0.000608640288262651
-0.532258064516129 0.000996293911671293
-0.516129032258065 0.00160703522774764
-0.5 0.00255431502841547
-0.483870967741935 0.00400068779255464
-0.467741935483871 0.0061745598471698
-0.451612903225806 0.00939049506239872
-0.435483870967742 0.0140728517890898
-0.419354838709677 0.0207819779603037
-0.403225806451613 0.0302414622728075
-0.387096774193548 0.0433640498261506
-0.370967741935484 0.0612728418578806
-0.354838709677419 0.0853134202261683
-0.338709677419355 0.117051726483672
-0.32258064516129 0.158252083057428
-0.306451612903226 0.210829892685455
-0.290322580645161 0.27677450389853
-0.274193548387097 0.358039642809822
-0.258064516129032 0.456401738767525
-0.241935483870968 0.573290316725243
-0.225806451612903 0.709599111776468
-0.209677419354839 0.86549120903068
-0.193548387096774 1.04021568904256
-0.17741935483871 1.23195623349442
-0.161290322580645 1.43773319445163
-0.145161290322581 1.65337916769813
-0.129032258064516 1.87360382182114
-0.112903225806452 2.09215668698248
-0.0967741935483871 2.30208731062605
-0.0806451612903226 2.49609158412319
-0.0645161290322581 2.66692241735952
-0.0483870967741936 2.80783374260418
-0.032258064516129 2.91302047046425
-0.0161290322580645 2.97801461768947
0 3
0.0161290322580645 2.97801461768947
0.032258064516129 2.91302047046425
0.0483870967741935 2.80783374260418
0.064516129032258 2.66692241735952
0.0806451612903225 2.49609158412319
0.096774193548387 2.30208731062605
0.112903225806452 2.09215668698248
0.129032258064516 1.87360382182114
0.145161290322581 1.65337916769814
0.161290322580645 1.43773319445164
0.17741935483871 1.23195623349442
0.193548387096774 1.04021568904256
0.209677419354839 0.86549120903068
0.225806451612903 0.709599111776468
0.241935483870968 0.573290316725243
0.258064516129032 0.456401738767525
0.274193548387097 0.358039642809822
0.290322580645161 0.27677450389853
0.306451612903226 0.210829892685456
0.32258064516129 0.158252083057428
0.338709677419355 0.117051726483672
0.354838709677419 0.0853134202261684
0.370967741935484 0.0612728418578806
0.387096774193548 0.0433640498261506
0.403225806451613 0.0302414622728076
0.419354838709677 0.0207819779603037
0.435483870967742 0.0140728517890898
0.451612903225806 0.00939049506239872
0.467741935483871 0.0061745598471698
0.483870967741935 0.00400068779255464
0.5 0.00255431502841547
0.516129032258065 0.00160703522774764
0.532258064516129 0.000996293911671293
0.548387096774194 0.000608640288262651
0.564516129032258 0.000366391219863391
0.580645161290323 0.000217340448944149
0.596774193548387 0.000127041965076447
0.612903225806452 7.3175380038363e-05
0.629032258064516 4.15330582707995e-05
0.645161290322581 2.3229185229476e-05
0.661290322580645 1.2802216980339e-05
0.67741935483871 6.9526045948028e-06
0.693548387096774 3.72066885997466e-06
0.709677419354839 1.96203002335803e-06
0.725806451612903 1.01953330858841e-06
0.741935483870968 5.22045468136291e-07
0.758064516129032 2.63406435887539e-07
0.774193548387097 1.30965098794868e-07
0.790322580645161 6.41646616757949e-08
0.806451612903226 3.09775725928519e-08
0.82258064516129 1.47370310643494e-08
0.838709677419355 6.90849989736991e-09
0.854838709677419 3.19130750008354e-09
0.870967741935484 1.45266238519863e-09
0.887096774193548 6.51586188929296e-10
0.903225806451613 2.87998476594066e-10
0.919354838709677 1.25435264960147e-10
0.935483870967742 5.38344466533261e-11
0.951612903225806 2.27673244330885e-11
0.967741935483871 9.48800445603038e-12
0.983870967741935 3.89626874434316e-12
1 1.57664555280194e-12
};
\addlegendentry{$\lambda$ = 3}
\addplot [semithick, white!46.6666666666667!black]
table {%
-1 5.91613846384715e-22
-0.983870967741935 2.95497350426647e-21
-0.967741935483871 1.43784105741592e-20
-0.951612903225806 6.81569579801307e-20
-0.935483870967742 3.14739713104568e-19
-0.919354838709677 1.41590761990003e-18
-0.903225806451613 6.20526483639392e-18
-0.887096774193548 2.64927933883491e-17
-0.870967741935484 1.10188738666265e-16
-0.854838709677419 4.46466316403702e-16
-0.838709677419355 1.76230963201255e-15
-0.82258064516129 6.77669217860779e-15
-0.806451612903226 2.53860616946365e-14
-0.790322580645161 9.26434952433317e-14
-0.774193548387097 3.29364327743847e-13
-0.758064516129032 1.14072305106815e-12
-0.741935483870968 3.84880495458687e-12
-0.725806451612903 1.26506726080405e-11
-0.709677419354839 4.05082385176855e-11
-0.693548387096774 1.26361603588392e-10
-0.67741935483871 3.83997968792083e-10
-0.661290322580645 1.1368018032873e-09
-0.645161290322581 3.27855602043547e-09
-0.629032258064516 9.21133319649312e-09
-0.612903225806452 2.52118319472481e-08
-0.596774193548387 6.72246180576739e-08
-0.580645161290323 1.74620131726489e-07
-0.564516129032258 4.41877926928207e-07
-0.548387096774194 1.08931211178044e-06
-0.532258064516129 2.61604003114404e-06
-0.516129032258065 6.12038111640634e-06
-0.5 1.3949369424836e-05
-0.483870967741935 3.09722469726345e-05
-0.467741935483871 6.6993532708667e-05
-0.451612903225806 0.00014116761384527
-0.435483870967742 0.000289787265258865
-0.419354838709677 0.000579516155070107
-0.403225806451613 0.00112899969938479
-0.387096774193548 0.00214271363699685
-0.370967741935484 0.00396165332923583
-0.354838709677419 0.00713560552544499
-0.338709677419355 0.0125206600831983
-0.32258064516129 0.0214025561794675
-0.306451612903226 0.0356406892662143
-0.290322580645161 0.0578187331015341
-0.274193548387097 0.0913761964710405
-0.258064516129032 0.14068234865141
-0.241935483870968 0.211002777409904
-0.225806451612903 0.308303717874142
-0.209677419354839 0.438845228098776
-0.193548387096774 0.608535651690032
-0.17741935483871 0.822058457442337
-0.161290322580645 1.08183593936955
-0.145161290322581 1.38695425205675
-0.129032258064516 1.73222736861949
-0.112903225806452 2.10760725071438
-0.0967741935483871 2.49813842909485
-0.0806451612903226 2.88459830992652
-0.0645161290322581 3.24486189528811
-0.0483870967741936 3.55589655647936
-0.032258064516129 3.79615597172062
-0.0161290322580645 3.94803510322454
0 4
0.0161290322580645 3.94803510322454
0.032258064516129 3.79615597172062
0.0483870967741935 3.55589655647936
0.064516129032258 3.24486189528811
0.0806451612903225 2.88459830992652
0.096774193548387 2.49813842909485
0.112903225806452 2.10760725071438
0.129032258064516 1.73222736861949
0.145161290322581 1.38695425205675
0.161290322580645 1.08183593936955
0.17741935483871 0.822058457442337
0.193548387096774 0.608535651690032
0.209677419354839 0.438845228098776
0.225806451612903 0.308303717874142
0.241935483870968 0.211002777409904
0.258064516129032 0.14068234865141
0.274193548387097 0.0913761964710405
0.290322580645161 0.0578187331015341
0.306451612903226 0.0356406892662144
0.32258064516129 0.0214025561794676
0.338709677419355 0.0125206600831983
0.354838709677419 0.00713560552544502
0.370967741935484 0.00396165332923584
0.387096774193548 0.00214271363699686
0.403225806451613 0.00112899969938479
0.419354838709677 0.000579516155070107
0.435483870967742 0.000289787265258865
0.451612903225806 0.00014116761384527
0.467741935483871 6.6993532708667e-05
0.483870967741935 3.09722469726345e-05
0.5 1.3949369424836e-05
0.516129032258065 6.12038111640634e-06
0.532258064516129 2.61604003114404e-06
0.548387096774194 1.08931211178044e-06
0.564516129032258 4.41877926928207e-07
0.580645161290323 1.7462013172649e-07
0.596774193548387 6.72246180576747e-08
0.612903225806452 2.52118319472482e-08
0.629032258064516 9.21133319649322e-09
0.645161290322581 3.2785560204355e-09
0.661290322580645 1.13680180328732e-09
0.67741935483871 3.83997968792088e-10
0.693548387096774 1.26361603588392e-10
0.709677419354839 4.05082385176855e-11
0.725806451612903 1.26506726080405e-11
0.741935483870968 3.84880495458687e-12
0.758064516129032 1.14072305106815e-12
0.774193548387097 3.29364327743847e-13
0.790322580645161 9.26434952433317e-14
0.806451612903226 2.53860616946365e-14
0.82258064516129 6.77669217860779e-15
0.838709677419355 1.76230963201256e-15
0.854838709677419 4.46466316403702e-16
0.870967741935484 1.10188738666266e-16
0.887096774193548 2.64927933883495e-17
0.903225806451613 6.20526483639397e-18
0.919354838709677 1.41590761990004e-18
0.935483870967742 3.14739713104568e-19
0.951612903225806 6.81569579801307e-20
0.967741935483871 1.43784105741592e-20
0.983870967741935 2.95497350426647e-21
1 5.91613846384715e-22
};
\addlegendentry{$\lambda$ = 4}
\end{axis}

\end{tikzpicture}

\end{tikzpicture}

Таким образом видно, что с ростом $\lambda$ дисперсия уменьшается.

\[\E(X)=0 \Rightarrow (\E(X))^2=0 \Rightarrow \sigma^2=\Var(X)=\E(X^2) \]
\[\sigma^2 =   \int_{-\infty}^{\infty} x^2 f(x) dx = 
  \int_{-\infty}^{\infty} x^2 \lambda e^{-\pi{\lambda}^2x^2}  dx =
	\lambda   \int_{-\infty}^{\infty} x\cdot x \cdot e^{-\pi{\lambda}^2x^2} dx =\]
\[ 
\left[\begin{matrix} u=x \quad \quad \quad \quad \quad du=dx\\ dv= x e^{-\pi{\lambda}^2x^2} dx \quad v=\frac{-1}{2\pi\lambda^2}e^{-\pi{\lambda}^2x^2}\end{matrix}\right]
\]
\[=\lambda \left( uv \bigg|_{-\infty}^{\infty}-   \int_{-\infty}^{\infty} vdu \right) = \lambda \left( -\frac{x}{2\pi\lambda^2}e^{-\pi{\lambda}^2x^2} \bigg|_{-\infty}^{\infty} +   \int_{-\infty}^{\infty} \frac{1}{2\pi\lambda^2}e^{-\pi{\lambda}^2x^2} dx \right)\]
Так как $e^{x^2}$ растёт быстрее, чем $x$ при $x \rightarrow \pm \infty$
\[\lim\limits_{x\to \pm \infty} \frac{x}{2\pi\lambda^2}e^{-\pi{\lambda}^2x^2} = 
\frac{1}{2\pi\lambda^2}\lim\limits_{x\to\infty} \frac{x}{ e^{-\pi{\lambda}^2x^2}}=0\]
Тогда:
\[\sigma^2 = \lambda  \int_{-\infty}^{\infty} \frac{1}{2\pi\lambda^2}e^{-\pi{\lambda}^2x^2} dx = \frac{1}{2\pi\lambda^2} \]
\[\lambda^2 = \frac{1}{2\pi\sigma^2} \Rightarrow \lambda = \frac{1}{\sqrt{2\pi\sigma^2}} \]
\[\Rightarrow f_X(x)=\frac{1}{\sqrt{2\pi\sigma^2}} e ^ {-\frac{x^2}{2\sigma^2}}, X \sim \mathcal{N}(0, \sigma^2)\]

Пусть $Z=X-\mu$, тогда:
\vspace{2cm}
\definecolor{aurometalsaurus}{rgb}{0.43, 0.5, 0.5}
\definecolor{bananayellow}{rgb}{1.0, 0.88, 0.21}
\definecolor{canaryyellow}{rgb}{1.0, 0.94, 0.0}
\begin{tcolorbox}[colback = canaryyellow]

\\
%\textcolor{white}{
\begin{huge}
\vspace{1cm}
\[f(z)=f(x-\mu) = \frac{1}{\sqrt{2\pi\sigma^2}} e ^ {-\frac{(x-\mu)^2}{2\sigma^2}} \sim \mathcal{N}(\mu, \sigma^2)\]
\vspace{0.4 in} 
\end{huge}
%}
\\
\end{tcolorbox}
\vspace{2cm}

\begin{tcolorbox}[colback = beaublue]

\begin{center}
    \textbf{График зависимости $f_X(x)$ от $\mu$}
\end{center}
\end{tcolorbox}
\vspace{2.5cm}
%\resizebox{25cm}{15cm}{	
	\begin{tikzpicture}
		% This file was created by tikzplotlib v0.9.2.
\begin{tikzpicture}

\definecolor{color0}{rgb}{0.886274509803922,0.290196078431373,0.2}
\definecolor{color1}{rgb}{0.203921568627451,0.541176470588235,0.741176470588235}
\definecolor{color2}{rgb}{0.596078431372549,0.556862745098039,0.835294117647059}

\begin{axis}[
axis background/.style={fill=white!89.8039215686275!black},
axis line style={white},
legend cell align={left},
legend style={fill opacity=0.8, draw opacity=1, text opacity=1, draw=white!80!black, fill=white!89.8039215686275!black},
tick align=outside,
tick pos=left,
title={\(\displaystyle f_X(x)=\frac{1}{\sqrt{2\pi\sigma^2}} e ^ {-\frac{(x-\mu)^2}{2\sigma^2}}\)},
x grid style={white},
xlabel={x},
xmajorgrids,
xmin=-5.5, xmax=5.5,
xtick style={color=white!33.3333333333333!black},
y grid style={white},
ylabel={\(\displaystyle f_X(x)\)},
ymajorgrids,
ymin=-0.0199379576580956, ymax=0.418697111020972,
ytick style={color=white!33.3333333333333!black}
]
\path [draw=black, fill=black, very thin]
(axis cs:1.3,0.38)
--(axis cs:1.25,0.375)
--(axis cs:1.25,0.3795)
--(axis cs:0.75,0.3795)
--(axis cs:0.75,0.3805)
--(axis cs:1.25,0.3805)
--(axis cs:1.25,0.385)
--cycle;
\path [draw=black, fill=black, very thin]
(axis cs:-1.3,0.38)
--(axis cs:-1.25,0.385)
--(axis cs:-1.25,0.3805)
--(axis cs:-0.75,0.3805)
--(axis cs:-0.75,0.3795)
--(axis cs:-1.25,0.3795)
--(axis cs:-1.25,0.375)
--cycle;
\addplot [thick, color0]
table {%
-5 1.4867195147343e-06
-4.8989898989899 2.45106104294233e-06
-4.7979797979798 3.99989037241663e-06
-4.6969696969697 6.4611663926488e-06
-4.5959595959596 1.03310065813217e-05
-4.49494949494949 1.63509588958253e-05
-4.39393939393939 2.56160811951383e-05
-4.29292929292929 3.97238223814148e-05
-4.19191919191919 6.09759039529897e-05
-4.09090909090909 9.26476353230142e-05
-3.98989898989899 0.000139341123134969
-3.88888888888889 0.000207440308767921
-3.78787878787879 0.000305686225478055
-3.68686868686869 0.000445889724807599
-3.58585858585859 0.000643795497926865
-3.48484848484848 0.000920104769622966
-3.38383838383838 0.00130165384164891
-3.28282828282828 0.00182273109600128
-3.18181818181818 0.00252649577810393
-3.08080808080808 0.00346643791897576
-2.97979797979798 0.00470779076312334
-2.87878787878788 0.00632877642858276
-2.77777777777778 0.00842153448411835
-2.67676767676768 0.0110925548393749
-2.57575757575758 0.0144624147976342
-2.47474747474747 0.0186646099340664
-2.37373737373737 0.0238432745020429
-2.27272727272727 0.0301496139168006
-2.17171717171717 0.0377369231406399
-2.07070707070707 0.0467541424067055
-1.96969696969697 0.0573380051248129
-1.86868686868687 0.0696039583923258
-1.76767676767677 0.0836361772145172
-1.66666666666667 0.0994771387927487
-1.56565656565657 0.117117359532744
-1.46464646464646 0.13648600918747
-1.36363636363636 0.157443187618844
-1.26262626262626 0.179774665124813
-1.16161616161616 0.203189835501224
-1.06060606060606 0.227323505631361
-0.959595959595959 0.251741946984239
-0.858585858585859 0.275953371470115
-0.757575757575758 0.2994226832711
-0.656565656565657 0.32159002340941
-0.555555555555555 0.341892294166129
-0.454545454545455 0.359786557812623
-0.353535353535354 0.37477397940639
-0.252525252525253 0.386422853089569
-0.151515151515151 0.394389234004919
-0.0505050505050502 0.398433801691346
0.0505050505050502 0.398433801691346
0.151515151515151 0.394389234004919
0.252525252525253 0.386422853089569
0.353535353535354 0.37477397940639
0.454545454545454 0.359786557812623
0.555555555555555 0.341892294166129
0.656565656565657 0.32159002340941
0.757575757575758 0.2994226832711
0.858585858585858 0.275953371470116
0.959595959595959 0.251741946984239
1.06060606060606 0.227323505631361
1.16161616161616 0.203189835501224
1.26262626262626 0.179774665124813
1.36363636363636 0.157443187618844
1.46464646464646 0.13648600918747
1.56565656565657 0.117117359532744
1.66666666666667 0.0994771387927487
1.76767676767677 0.0836361772145173
1.86868686868687 0.0696039583923258
1.96969696969697 0.0573380051248129
2.07070707070707 0.0467541424067055
2.17171717171717 0.03773692314064
2.27272727272727 0.0301496139168007
2.37373737373737 0.0238432745020429
2.47474747474747 0.0186646099340664
2.57575757575758 0.0144624147976342
2.67676767676768 0.0110925548393749
2.77777777777778 0.00842153448411835
2.87878787878788 0.00632877642858276
2.97979797979798 0.00470779076312334
3.08080808080808 0.00346643791897576
3.18181818181818 0.00252649577810393
3.28282828282828 0.00182273109600128
3.38383838383838 0.00130165384164891
3.48484848484848 0.000920104769622967
3.58585858585859 0.000643795497926863
3.68686868686869 0.000445889724807599
3.78787878787879 0.000305686225478057
3.88888888888889 0.00020744030876792
3.98989898989899 0.000139341123134969
4.09090909090909 9.26476353230145e-05
4.19191919191919 6.09759039529897e-05
4.29292929292929 3.9723822381415e-05
4.39393939393939 2.56160811951381e-05
4.49494949494949 1.63509588958253e-05
4.5959595959596 1.03310065813217e-05
4.6969696969697 6.4611663926488e-06
4.7979797979798 3.99989037241663e-06
4.8989898989899 2.45106104294232e-06
5 1.4867195147343e-06
};
\addlegendentry{$\mu = 0$}
\addplot [thick, color1]
table {%
-5 9.13472040836459e-12
-4.8989898989899 1.84313269683253e-11
-4.7979797979798 3.68117804861537e-11
-4.6969696969697 7.2775620920913e-11
-4.5959595959596 1.42414395655991e-10
-4.49494949494949 2.75861274780737e-10
-4.39393939393939 5.28927883267668e-10
-4.29292929292929 1.00385517297334e-09
-4.19191919191919 1.8858820907808e-09
-4.09090909090909 3.50692830265765e-09
-3.98989898989899 6.45517638429505e-09
-3.88888888888889 1.17613782167113e-08
-3.78787878787879 2.12117839118739e-08
-3.68686868686869 3.78673617262774e-08
-3.58585858585859 6.69147374185849e-08
-3.48484848484848 1.17043522507282e-07
-3.38383838383838 2.02647786875826e-07
-3.28282828282828 3.47300351143572e-07
-3.18181818181818 5.89165700391928e-07
-3.08080808080808 9.89324044645218e-07
-2.97979797979798 1.64440409278185e-06
-2.87878787878788 2.70549919796867e-06
-2.77777777777778 4.40610807815217e-06
-2.67676767676768 7.10283552138762e-06
-2.57575757575758 1.13338443961373e-05
-2.47474747474747 1.79015901371739e-05
-2.37373737373737 2.79881922967599e-05
-2.27272727272727 4.33138675815426e-05
-2.17171717171717 6.63510737893166e-05
-2.07070707070707 0.000100609229401117
-1.96969696969697 0.000151006821176501
-1.86868686868687 0.000224349023712252
-1.76767676767677 0.000329929142567962
-1.66666666666667 0.000480270651620821
-1.56565656565657 0.000692022626719546
-1.46464646464646 0.000987014235438332
-1.36363636363636 0.00139346291172348
-1.26262626262626 0.00194731535113808
-1.16161616161616 0.00269368021718555
-1.06060606060606 0.00368828660920773
-0.959595959595959 0.00499887368345295
-0.858585858585859 0.00670638588963139
-0.757575757575758 0.00890581751341603
-0.656565656565657 0.011706522931417
-0.555555555555555 0.0152317893271017
-0.454545454545455 0.0196174612857474
-0.353535353535354 0.0250094165414517
-0.252525252525253 0.0315597236355863
-0.151515151515151 0.0394213687040283
-0.0505050505050502 0.0487415215163015
0.0505050505050502 0.0596534191274479
0.151515151515151 0.0722670747823327
0.252525252525253 0.0866591622561734
0.353535353535354 0.102862570441977
0.454545454545454 0.120856255671499
0.555555555555555 0.140556124161394
0.656565656565657 0.16180773804248
0.757575757575758 0.184381641233662
0.858585858585858 0.207972035232983
0.959595959595959 0.232199394625005
1.06060606060606 0.256617399843244
1.16161616161616 0.280724290633315
1.26262626262626 0.303978425912008
1.36363636363636 0.32581749943769
1.46464646464646 0.34568053588946
1.56565656565657 0.363031510505906
1.66666666666667 0.377383227692993
1.76767676767677 0.388319985112051
1.86868686868687 0.395517556374466
1.96969696969697 0.398759153353742
2.07070707070707 0.397946271863327
2.17171717171717 0.393103663637676
2.27272727272727 0.384378084457259
2.37373737373737 0.372030906761215
2.47474747474747 0.356425115605715
2.57575757575758 0.338007590643616
2.67676767676768 0.317287880332747
2.77777777777778 0.294814872934877
2.87878787878788 0.271152848320478
2.97979797979798 0.246858353660005
3.08080808080808 0.222459195106374
3.18181818181818 0.198436596963085
3.28282828282828 0.175211277227103
3.38383838383838 0.153133855197058
3.48484848484848 0.132479674584754
3.58585858585859 0.11344782261826
3.68686868686869 0.0961638745203805
3.78787878787879 0.0806857085231638
3.88888888888889 0.0670116261184602
3.98989898989899 0.0550899746196203
4.09090909090909 0.0448294967585928
4.19191919191919 0.0361097125126886
4.29292929292929 0.0287907563222775
4.39393939393939 0.0227222321450954
4.49494949494949 0.0177507941293637
4.5959595959596 0.0137262990840641
4.6969696969697 0.0105064985221502
4.7979797979798 0.00796033648403002
4.8989898989899 0.00596999164887265
5 0.00443184841193801
};
\addlegendentry{$\mu = \mu_0 = 2$}
\addplot [thick, color2]
table {%
-5 0.00443184841193801
-4.8989898989899 0.00596999164887267
-4.7979797979798 0.00796033648403002
-4.6969696969697 0.0105064985221502
-4.5959595959596 0.0137262990840641
-4.49494949494949 0.0177507941293637
-4.39393939393939 0.0227222321450954
-4.29292929292929 0.0287907563222774
-4.19191919191919 0.0361097125126886
-4.09090909090909 0.0448294967585927
-3.98989898989899 0.0550899746196203
-3.88888888888889 0.0670116261184602
-3.78787878787879 0.0806857085231636
-3.68686868686869 0.0961638745203805
-3.58585858585859 0.11344782261826
-3.48484848484848 0.132479674584754
-3.38383838383838 0.153133855197058
-3.28282828282828 0.175211277227103
-3.18181818181818 0.198436596963085
-3.08080808080808 0.222459195106374
-2.97979797979798 0.246858353660005
-2.87878787878788 0.271152848320478
-2.77777777777778 0.294814872934877
-2.67676767676768 0.317287880332747
-2.57575757575758 0.338007590643616
-2.47474747474747 0.356425115605715
-2.37373737373737 0.372030906761215
-2.27272727272727 0.384378084457259
-2.17171717171717 0.393103663637676
-2.07070707070707 0.397946271863327
-1.96969696969697 0.398759153353742
-1.86868686868687 0.395517556374466
-1.76767676767677 0.388319985112051
-1.66666666666667 0.377383227692993
-1.56565656565657 0.363031510505906
-1.46464646464646 0.34568053588946
-1.36363636363636 0.32581749943769
-1.26262626262626 0.303978425912008
-1.16161616161616 0.280724290633315
-1.06060606060606 0.256617399843244
-0.959595959595959 0.232199394625005
-0.858585858585859 0.207972035232984
-0.757575757575758 0.184381641233662
-0.656565656565657 0.16180773804248
-0.555555555555555 0.140556124161394
-0.454545454545455 0.120856255671499
-0.353535353535354 0.102862570441977
-0.252525252525253 0.0866591622561734
-0.151515151515151 0.0722670747823327
-0.0505050505050502 0.0596534191274479
0.0505050505050502 0.0487415215163015
0.151515151515151 0.0394213687040283
0.252525252525253 0.0315597236355863
0.353535353535354 0.0250094165414517
0.454545454545454 0.0196174612857474
0.555555555555555 0.0152317893271017
0.656565656565657 0.011706522931417
0.757575757575758 0.00890581751341603
0.858585858585858 0.00670638588963141
0.959595959595959 0.00499887368345295
1.06060606060606 0.00368828660920773
1.16161616161616 0.00269368021718555
1.26262626262626 0.00194731535113809
1.36363636363636 0.00139346291172348
1.46464646464646 0.000987014235438332
1.56565656565657 0.000692022626719546
1.66666666666667 0.00048027065162082
1.76767676767677 0.000329929142567963
1.86868686868687 0.000224349023712252
1.96969696969697 0.000151006821176501
2.07070707070707 0.000100609229401117
2.17171717171717 6.63510737893166e-05
2.27272727272727 4.33138675815428e-05
2.37373737373737 2.79881922967599e-05
2.47474747474747 1.79015901371739e-05
2.57575757575758 1.13338443961373e-05
2.67676767676768 7.10283552138762e-06
2.77777777777778 4.40610807815217e-06
2.87878787878788 2.70549919796867e-06
2.97979797979798 1.64440409278185e-06
3.08080808080808 9.89324044645218e-07
3.18181818181818 5.89165700391928e-07
3.28282828282828 3.47300351143574e-07
3.38383838383838 2.02647786875826e-07
3.48484848484848 1.17043522507282e-07
3.58585858585859 6.69147374185846e-08
3.68686868686869 3.78673617262774e-08
3.78787878787879 2.1211783911874e-08
3.88888888888889 1.17613782167113e-08
3.98989898989899 6.45517638429505e-09
4.09090909090909 3.50692830265767e-09
4.19191919191919 1.8858820907808e-09
4.29292929292929 1.00385517297335e-09
4.39393939393939 5.28927883267664e-10
4.49494949494949 2.75861274780737e-10
4.5959595959596 1.42414395655992e-10
4.6969696969697 7.2775620920913e-11
4.7979797979798 3.68117804861537e-11
4.8989898989899 1.84313269683252e-11
5 9.13472040836459e-12
};
\addlegendentry{$\mu = \mu_1 = -2$}
\end{axis}

\end{tikzpicture}

	\end{tikzpicture}
	
%}




\end{column}

\begin{column}{.015\textwidth}\end{column} % Empty spacer column

\end{columns} % End of all the columns in the poster

\end{frame} % End of the enclosing frame

\end{document}